% SPDX-License-Identifier: GPL-3.0-or-later OR CC-BY-SA-4.0
\section{ADB over TCP}\label{sec:faq:adb-over-tcp} %%##$section-title>>

\subsection{Do I have to enable ADB over TCP everytime I restart?}\label{subsec:faq:enable-adb-on-every-restart} %%##$restart-title>>
%%!!restart<<
Unfortunately, yes. This is because the ADB daemon, the process responsible for ADB connection, is also restarted after
a reboot, and it does not re-enable ADB over TCP\@.
%%!!>>

\subsection{Cannot enable USB debugging. What to do?}\label{subsec:faq:usb-debugging} %%##$usb-debugging-title>>
%%!!usb-debugging<<
See \Sref{subsec:enable-usb-debugging} in \Cref{ch:guides}.
%%!!>>

\subsection{Can I block tracker or any other application components using ADB over TCP?}\label{subsec:faq:block-components-using-adb} %%##$block-tracker-title>>
%%!!block-tracker<<
ADB has limited number of \href{https://github.com/aosp-mirror/platform_frameworks_base/blob/master/packages/Shell/AndroidManifest.xml}{permissions}
and controlling application components is not one of them. However, the components of a \textit{test-only} app can be
controlled via ADB\@. If App Manager detects such an application, it enables the blocking options automatically.
%%!!>>

\subsection{Which features can be used in ADB mode?}\label{subsec:faq:adb-features} %%##$feature-adb-title>>
%%!!feature-adb<<
Supported features are enabled automatically in the ADB mode. Supported features include disabling, force-stopping,
clearing application data, granting or revoking app ops and permissions, and so on. It is also possible to install or
uninstall applications without any prompt from the system.
%%!!>>
