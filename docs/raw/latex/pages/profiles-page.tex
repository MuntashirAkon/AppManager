\section{Profiles Page}\label{sec:profiles-page} %%##pages_profilespage-section==title>>
%%!!pages_profilespage-section<<
Profiles page can be accessed from the options-menu in the main page. It displays a list of configured profiles.
Profiles can be added using the \textit{plus} button at the bottom-right corner, imported from the import option,
created from one of the presets or even duplicated from an already existing profile. Clicking on any profile opens the
\hyperref[sec:profile-page]{profile page}.
%%!!>>

\subsection{Options Menu}\label{subsec:profiles-options-menu} %%##pages_profilespage-profiles-options-menu==title>>
%%!!pages_profilespage-profiles-options-menu<<
There are two options menu in this page. The three dots menu at the top-right offers two options such as
\textit{presets} and \textit{import}.
\begin{itemize}
    \item \textbf{Presets.} Presets option lists a number of built-in profiles that can be used as a starting point.
    The profiles are generated from the project \href{https://gitlab.com/W1nst0n/universal-android-debloater}{Universal
    Android Debloater}.\\
    \seealsoinline{\hyperref[subsec:faq:what-are-bloatware]{FAQ: What are bloatware and how to remove them?}}

    \item \textbf{Import.} This option can be used to import an existing profile.
\end{itemize}

Another options menu appears when you long click on any profile. They have options such as--
\begin{itemize}
    \item \textbf{Apply now\dots.} This option can be used to apply the profile directly. When clicked, a dialog will be
    displayed where you can select a \hyperref[subsubsec:profile-state]{profile state}. On selecting one of the options,
    the profile will be applied immediately.
    \item \textbf{Delete.} Clicking on delete will remove the profile immediately without any warning.
    \item \textbf{Duplicate.} This option can be used to duplicate the profile. When clicked, an input box will be
    displayed where you can set the profile name. If you click ``OK'', a new profile will be created and the
    \hyperref[sec:profile-page]{profile page} will be loaded. The profile will not be saved until you save it manually.
    \item \textbf{Export.} Export the profile to an external storage. Profile exported this way can be imported using
    the \textit{import} option in the three dots menu.
    \item \textbf{Create shortcut.} This option can be used to create a shortcut for the profile. When clicked, there
    will be two options: \textit{Simple} and \textit{Advanced}. The latter option allows you to set the profile state
    before applying it while the former option use the default state that was configured when the profile was last saved.
\end{itemize}
%%!!>>
