% SPDX-License-Identifier: GPL-3.0-or-later OR CC-BY-SA-4.0
\chapter{Introduction}\label{ch:introduction} %%##intro_main-introduction==title>>
%%!!intro_main-introduction<<

App Manager is an advanced package manager for Android. It offers many features and consequently, it requires a user
manual to assist its users with features that it offers. This document acts as a user manual for App Manager in the
sense that it contains the description of every feature that App Manager has to offer. The document can be considered as
the ``official'' guidelines for App Manager and represent the expected behaviour of App Manager. Translations can
misinterpret this document (which is written in English). Therefore, it is recommended that you read the English version
if possible. There might be other unofficial or third-party resources such as blog articles, videos, chat groups, etc.
While these resources might be useful for many people, they may not be up-to-date. In case any deviations are detected
in App Manager from this document, they should be reported in the
\href{https://github.com/MuntashirAkon/AppManager/issues}{App Manager issue tracker}.
%%!!>>


\section{Terminologies}\label{sec:terminologies} %%##intro_main-terminologies==title>>
%%!!intro_main-terminologies<<
\begin{itemize}
    \item \textbf{AM} --- Short name for App Manager
    \item \textbf{AoT} --- Short form of ADB over TCP
    \item \textbf{Block/Unblock} --- Used for component blocking. It denotes that a component is blocked using Intent
    Firewall and disabled using PackageManager.
    \item \textbf{IFW} --- Short form of Intent Firewall.
    \item \textbf{Ops} --- Short name for operations, e.g.\ app ops, batch ops, 1-click ops
    \item \textbf{SSAID} --- Short form of \texttt{Settings.Secure.ANDROID\_ID}. It is a device identifier assigned to
    each app (from Android Oreo and onwards). It is generated from the combination of the signing certificate of the app
    and the SSAID set for the package \texttt{android}. As a result, it is guaranteed to be the same for an app unless
    the user formats the device. It is widely used for tracking users.
    \item \textbf{Tracker} --- Denotes tracker components throughout the document and in App Manager except the
    \hyperref[sec:scanner-page]{scanner page}. Trackers include libraries such as crash reporters, analytics,
    profiling, identification, ads, location, etc. Thus, they are not equal in functions. There is no distinction or
    bias between open source and closed source libraries.
\end{itemize}
%%!!>>

\section{Supported Versions}\label{sec:supported-versions} %%##intro_main-supportedversions==title>>
%%!!intro_main-supportedversions<<
Currently supported versions are v2.5.20 (stable), v2.5.23 (pre-release) and v2.5.24 (debug release).
%%!!>>

\section{Official Sources}\label{sec:official-sources} %%##intro_main-officialsources==title>>

\subsection{Binary Distribution Sources}\label{subsec:binary-distribution-sources} %%##intro_main-binsources==title>>
%%!!intro_main-binsources<<
App Manager is distributed using the following sources. Unofficial sources may distribute modified versions of App
Manager, and none but you shall be responsible for the consequences of using such distributions.
\begin{enumerate}
    \item Official F-Droid repository.\footnote{For distributing stable releases only}\\
    \textit{Link:} \url{https://f-droid.org/packages/io.github.muntashirakon.AppManager}
    \item IzzyOnDroid repository for F-Droid.\footnote{For distributing pre-releases only}\\
    \textit{Link:} \url{https://apt.izzysoft.de/fdroid/index/apk/io.github.muntashirakon.AppManager}
    \item GitHub repository.\\
    \textit{Normal releases:} \url{https://github.com/MuntashirAkon/AppManager/releases}\\
    \textit{Debug releases:} \url{https://github.com/MuntashirAkon/AppManager/actions}
    \item Telegram.\\
    \textit{Normal releases:} \url{https://t.me/AppManagerChannel}\\
    \textit{Debug releases:} \url{https://t.me/AppManagerDebug}
\end{enumerate}
%%!!>>

\subsection{Links to Source Code}\label{subsec:links-to-source-code} %%##intro_main-sourcecodelinks==title>>
%%!!intro_main-sourcecodelinks<<
All but GitHub are the mirror links. The tags should always be up-to-date, but the master branch is not guaranteed to be
up-to-date. If the target is to clone the master branch, use the GitHub link instead of the others.
%%!!>>
\begin{enumerate}
    \item GitHub: \url{https://github.com/MuntashirAkon/AppManager}
    \item GitLab: \url{https://gitlab.com/muntashir/AppManager}
    \item Codeberg: \url{https://codeberg.org/muntashir/AppManager}
    \item Riseup: \url{https://0xacab.org/muntashir/AppManager}
\end{enumerate}

\subsection{Translations}\label{subsec:translations} %%##intro_main-translations==title>>
%%!!intro_main-translations<<
App Manager does not accept translations directly via pull/merge requests. Translations are managed both automatically
and manually depending on the context.
\begin{itemize}
    \item App Manager app strings are hosted in Weblate.\\
    \textit{Link:} \url{https://hosted.weblate.org/engage/app-manager/}
    \item App Manager user manual are hosted in Crowdin.\\
    \textit{Link:} \url{https://crwd.in/app-manager-docs}
\end{itemize}
%%!!>>

\section{Contributing}\label{sec:contributing} %%##intro_main-contributing==title>>
%%!!intro_main-contributing<<
There are multiple ways a user can contribute such as creating helpful issues, attending discussions, improving
documentations and translations, adding unrecognised libraries or trackers, reviewing the source code as well as
reporting security vulnerabilities.
%%!!>>

\subsection{Build Instructions}\label{sec:build-instructions} %%##intro_main-buiding==title>>
%%!!intro_main-buiding<<
Build instructions are available in the README file of the root directory of the source.
%%!!>>

\subsection{Submitting patches}\label{subsec:submitting-patches} %%##intro_main-submitpatches==title>>
%%!!intro_main-submitpatches<<
Repositories located in sites other than GitHub are currently considered mirrors and pull/merge requests submitted there
will not be accepted. Instead, patches (as .patch files) can be submitted via email attachments. Make sure to sign-off
your commits.

\begin{tip}{Info}
    GitHub PRs will be merged manually using the corresponding patches. As a result, GitHub may falsely mark them closed
    instead of merged.
\end{tip}

\begin{warning}{Notice}
    In case of submitting patches via email, the whole conversation may be publicly accessible in the future. So, do not
    include any personally identifiable information (PII) other than your name or email address.
\end{warning}
%%!!>>

\section{Donation \& Funding}\label{sec:donation-&-funding} %%##intro_main-donation==title>>
%%!!intro_main-donation_1<<
\emph{Donation or purchase is not a requirement in order to use App Manager.} While App Manager does not support any
purchases, donations can be sent to the author of App Manager using one of the following Bitcoin addresses as a tribute:
%%!!>>
\begin{Verbatim}[label={verb:btc-donation}]
    33TDkWVv5EgwfKGJk7YaS2Ev1CBzBP9Sav
    38bzvWDD99dJhXg9tC4yQEnGdnAKPtwSXG
    3FHTxPoYa92dNJK6pkhwyVkMG8Vv3VpGpg
\end{Verbatim}

%%!!intro_main-donation_2<<
By sending BTC, the sender agrees that he/she/they shall not make the transaction information public i.e.\ the
transaction will remain anonymous, nor shall he/she/they use it as a leverage to prioritise his/her/their requested
features. Feature requests do not require any bounties or donations, and they are prioritised according to the authors'
preferences.

\emph{App Manager is open for funding/grants.} Representatives of the interested organizations can contact the author
directly using the options given in §\cref{sec:contact}.
%%!!>>

\section{Contact}\label{sec:contact} %%##intro_main-contact==title>>
%%!!intro_main-contact<<
Muntashir Al-Islam\footnote{You can also address me as ``Muntashir Akon''}\\
Email: \href{mailto:muntashirakon@riseup.net}{muntashirakon [at] riseup [dot] net}\\
Key Fingerprint: \texttt{7bad37c2981e41f8f6abea7f58f0b4f26c346fce}\\
GitHub: \url{https://github.com/MuntashirAkon}\\
Twitter: \url{https://twitter.com/Muntashir}
%%!!>>

